\documentclass[final,leqno]{article}

\pagestyle{fancy}
\oddsidemargin=-0.5in 
\evensidemargin=-0.5in
\textwidth=7.5in 
% \headwidth=7.5in
\textheight=9.0in 
\headheight=0.5pt
\topmargin=-0.5in
\headsep=0.5in


\newcommand{\pe}{\psi}
\def\d{\delta} 
\def\ds{\displaystyle} 
\def\e{{\epsilon}} 
\def\eb{\bar{\eta}}  
\def\enorm#1{\|#1\|_2} 
\def\Fp{F^\prime}  
\def\fishpack{{FISHPACK}} 
\def\fortran{{FORTRAN}} 
\def\gmres{{GMRES}} 
\def\gmresm{{\rm GMRES($m$)}} 
\def\Kc{{\cal K}} 
\def\norm#1{\|#1\|} 
\def\wb{{\bar w}} 
\def\zb{{\bar z}} 


\def\bfE{\mbox{\boldmath$E$}}
\def\bfG{\mbox{\boldmath$G$}}

\title{\vspace{-50pt}Statistical Connectomics}


\author{jovo\thanks{yummy}
        \and cep\thanks{t}}

\begin{document}

\maketitle
\tableofcontents

\begin{abstract}
% An example of SIAM \LaTeX\ macros is presented. Various
% aspects of composing manuscripts for SIAM's journal series
% are illustrated with actual examples from accepted
% manuscripts. SIAM's stylistic standards are adhered to
% throughout, and illustrated.
\end{abstract}

% \begin{keywords} 
% sign-nonsingular matrix, LU-factorization, indicator
% polynomial 
% \end{keywords}

% \begin{AMS}
% 15A15, 15A09, 15A23
% \end{AMS}

\pagestyle{myheadings}
\thispagestyle{plain}
% \markboth{TEX PRODUCTION AND V. A. U. THORS}{SIAM MACRO EXAMPLES}

\clearpage
\section{Introduction}

additional core authors: runze? 

potential additional graphstat authors: li? nam? youngser? daniele, dunson, vince,  minh, daniel

potential neuro co-authors could include: 
mike milham, scott cook, mitya, bobby/jeff, clay/davi, rex jung, 

we start by stating how important connectomics will be for the future of neuroscience, and how having rigorous statistical theory will enable future investigations to leverage it to substantiate their claims.

for each exploitation task, we provide: 
\begin{enumerate}
\item rigorous definition
\item motivating application
\item R code
\item images, graphs, and graph derivatives downloads
\end{enumerate}


\section{One Sample Tests}

\subsection{tests for independence between connectivity and vertex attributes (such as direction preference, excitatory vs. inhibitory, etc.)}

bock11 \cite{Bock2011} dataset, testing independence of tuning direction vs connectivity, using residual error of regression o ase as test statistic, permutation test to obtain null

\subsection{tests for independence between space and connectivity}

kasthuri11 dataset (no cite yet, coming soon), touches vs. synapses, using whatever we do (probably importance sampling to obtain null distribution)


\subsection{tests for model fit}

hsbm on fly optic lobe data \cite{Takemura2013}, likelihood test via parametric bootstrap

\section{2-sample tests for comparative connectomics}

\subsection{comparing 2 different connectomes}

elegans electrical vs. chemical \& elegans vs. pacificus \& elegans male vs. herm.  See \cite{Varshney2011} for the most clear description of these graphs.


\subsection{2 populations of connectomes}

\cite{Nooner2012,Landman2010} describes two different populations of subjects collected for two different studies, both of which are useful.

\section{population density estimation}

\subsection{mean estimation}

\cite{Johnson71a,Gutmann82a} are two papers proving that Stein's paradox does not occur in finite spaces, in other words, $\bar{A}$ is admissible under squared error loss.  nonetheless, it seems likely that some smoothing/regularizing of $\bar{A}$ would be advantageous for finite sample sizes.  in particular, spectral and constrained estimates of latent vectors.  we can use any number of MR datasets, such as those MRN-111 in \cite{MIGRAINE}.


\subsection{robust mean (eg, median, or Lq) estimation}

we can again use the MRN-111 dataset, the theory is motivated by \cite{Ferrari2010a,Qin2013a}.


\subsection{Clustering}

using tensor factorizations \cite{Lee13a,Lee14a,Lee14b}, or DELTACON \cite{Koutra2013,Koutra2014}, which is just hclust with a different dissimilarity function.

\subsection{errorbars around mean estimation, eg, estimation variance}
 
bayesian nonparametric model \cite{Durante14a}


\section{connectome coding}


\subsection{classifying connectomes}

signal subgraphs paper \cite{signal-subgraph}, or using ASE or tensor factorization, followed by classical classification.

\subsection{regressing connectomes}

MRN114 via NTF followed by regression onto CCI

\subsection{multivariate regression for connectomes}

Adelstein \cite{Adelstein2011a} using JoFC on 5-factor personality test.


\section{Discussion}

general issues:

\subsection{bias variance trade-off: num params $>$ num subjects}

\subsection{nuisance signals: age, sex, batch}

\subsection{Graph Matching}

oh, which papers to list, how about \cite{VP11_FAQ,Fishkind2012a,sgm-jofc,Lyzinski2013}

\subsection{Future Work}


\section*{Acknowledgments}
The author thanks the anonymous authors whose work largely
constitutes this sample file. He also thanks the INFO-TeX mailing
list for the valuable indirect assistance he received.
 
 
\bibliography{../../../../other/latex/library}
\bibliographystyle{IEEEtran}


\end{document} 

